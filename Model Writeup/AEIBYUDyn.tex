\documentclass[letterpaper,12pt]{article}
  \usepackage{threeparttable}
  \usepackage{geometry}
  \geometry{letterpaper,tmargin=1in,bmargin=1in,lmargin=1.25in,rmargin=1.25in}
  \usepackage[format=hang,font=normalsize,labelfont=bf]{caption}
  \usepackage{amsmath}
  \usepackage{multirow}
  \usepackage{array}
  \usepackage{delarray}
  \usepackage{amssymb}
  \usepackage{amsthm}
  \usepackage{lscape}
  \usepackage{natbib}
  \usepackage{setspace}
  \usepackage{float,color}
  \usepackage[pdftex]{graphicx}
  \usepackage{pdfsync}
  \usepackage{verbatim}
  \usepackage{placeins}
  \synctex=1
  \usepackage{hyperref}
  \hypersetup{colorlinks,linkcolor=red,urlcolor=blue,citecolor=red}
  \usepackage{bm}

  \theoremstyle{definition}
  \newtheorem{theorem}{Theorem}
  \newtheorem{acknowledgement}[theorem]{Acknowledgement}
  \newtheorem{algorithm}[theorem]{Algorithm}
  \newtheorem{axiom}[theorem]{Axiom}
  \newtheorem{case}[theorem]{Case}
  \newtheorem{claim}[theorem]{Claim}
  \newtheorem{conclusion}[theorem]{Conclusion}
  \newtheorem{condition}[theorem]{Condition}
  \newtheorem{conjecture}[theorem]{Conjecture}
  \newtheorem{corollary}[theorem]{Corollary}
  \newtheorem{criterion}[theorem]{Criterion}
  \newtheorem{definition}{Definition} % Number definitions on their own
  \newtheorem{derivation}{Derivation} % Number derivations on their own
  \newtheorem{example}[theorem]{Example}
  \newtheorem{exercise}[theorem]{Exercise}
  \newtheorem{lemma}[theorem]{Lemma}
  \newtheorem{notation}[theorem]{Notation}
  \newtheorem{problem}[theorem]{Problem}
  \newtheorem{proposition}{Proposition} % Number propositions on their own
  \newtheorem{remark}[theorem]{Remark}
  \newtheorem{solution}[theorem]{Solution}
  \newtheorem{summary}[theorem]{Summary}
  \bibliographystyle{aer}
  \newcommand\ve{\varepsilon}
  \newcommand{\cn}{\citeasnoun} % shortens command to cite as noun
  \renewcommand\theenumi{\roman{enumi}}
  \newcommand\norm[1]{\left\lVert#1\right\rVert}

\begin{document}

%titlepage
\begin{titlepage}
  \title{Dyanmic General Equilibrim Tax Scoring Model
    \thanks{We are grateful to Kevin Hassett, Alan Viard, Alex Brill, Matt Jensen, Aspen Gorry, Frank Caliendo, and Richard W. Evans, Sr. for helpful comments and suggestions. This research benefited from support from Brigham Young University Macroeconomics and Computational Laboratory and the Open Source Policy Center at the American Enterprise Institute. All Python code for the computational model is available at \href{https://github.com/OpenSourcePolicyCenter/dynamic}{https://github.com/OpenSourcePolicyCenter/dynamic}.} }

  \author{
  Jason DeBacker\footnote{Middle Tennessee State University, Department of Economics and Finance, BAS N306, Murfreesboro, TN 37132, (615) 898-2528,\href{mailto:jason.debacker@mtsu.edu}{jason.debacker@mtsu.edu}.} \\[-2pt]
  \and
  Richard W. Evans\footnote{Brigham Young University, Department of Economics, 167 FOB, Provo, Utah 84602, (801) 422-8303, \href{mailto:revans@byu.edu}{revans@byu.edu}.} \\[-2pt]
  \and
  Evan Magnusson\footnote{Brigham Young University, Department of Economics, 163 FOB, Provo, Utah 84602, \href{mailto:evanmag42@gmail.com}{evanmag42@gmail.com}.} \\[-2pt]
  \and
  Kerk L. Phillips\footnote{Brigham Young University, Department of Economics, 166 FOB, Provo, Utah 84602, (801) 422-5928, \href{mailto:kerk_phillips@byu.edu}{kerk\_phillips@byu.edu}.} \\[-2pt]
  \and
  Shanthi P. Ramnath\footnote{U.S. Department of the Treasury, Office of Tax Analysis, 1500 Pennsylvania Ave. NW, Washington, DC 20220, (202) 622-1789, \href{mailto:shanthi.ramnath@treasury.gov}{shanthi.ramnath@treasury.gov}.} \\[-2pt]
  \and
  Isaac Swift\footnote{Brigham Young University, Department of Economics, 163 FOB, Provo, Utah 84602, \href{mailto:isaacdswift@gmail.com}{isaacdswift@gmail.com}.} \\[-2pt]}
  \date{February 2015 \\
  \scriptsize{(version 15.02.a)}}
  \maketitle
  \begin{abstract}
  \small{This paper ...

  \vspace{0.3in}

  \textit{keywords:} dynamic general equilibrium, taxation, numerical simulation, computational techniques, simulation modeling.

  \vspace{0.3in}

  \textit{JEL classifications:} C63, C68, E62, H24, H25, H68}
  \end{abstract}
  \thispagestyle{empty}
\end{titlepage}

\begin{spacing}{1.5}


\section{Introduction}\label{Sec_Intro}

\section{Details of the Macro Model}\label{Sec_Macro}

  \subsection{Deomgraphics}
    A measure $\omega_{1,t}$ of individuals with heterogeneous working ability $e \in\mathcal{E}\subset\mathbb{R}_{++}$ is born in each period $t$ and live for $E+S$ periods, with $S\geq 4$.\footnote{Theoretically, the model exposition of the model works without loss of generality for $S\geq 3$. However, because we are calibrating the ages outside of the economy to be one-fourth of $S$ (e.g., ages 21 to 100 in the economy, and ages 1 to 20 outside of the economy), we need $S$ to be at least 4.} The population of age-$s$ individuals in any period $t$ is $\omega_{s,t}$. Households are termed ``youth'' and out of the market during ages $1\leq s\leq E$. The households enter the workforce and economy in period $E+1$ and remain in the workforce until they unexpectedly die or live until age $s=E+S$.\footnote{We model the population with households age $s\leq E$ outside of the workforce and economy in order most closely match the empirical population dynamics.} The population of agents of each age in each period $\omega_{s,t}$ evolves according to the following function,
    \begin{equation}\label{EqPopLawofmotion}
      \begin{split}
        \omega_{1,t+1} &= \sum_{s=1}^{E+S} f_s\omega_{s,t}\quad\forall t \\
        \omega_{s+1,t+1} &= (1 + i_s - \rho_s)\omega_{s,t}\quad\forall t\quad\text{and}\quad 1\leq s \leq E+S-1
      \end{split}
    \end{equation}
    where $f_s\geq 0$ is an age-specific fertility rate, $i_s$ is an age-specific immigration rate, $\rho_s$ is an age specific mortality hazard rate,\footnote{The parameter $\rho_s$ is the probability that a household of age $s$ dies before age $s+1$.} and $1+i_s-\rho_s$ is constrained to be nonnegative. The total population in the economy $N_t$ at any period is simply the sum of individuals in the economy, the population growth rate in any period $t$ from the previous period $t-1$ is $g_{n,t}$, $\tilde{N}_t$ is the working age population, and $\tilde{g}_{n,t}$ is the working age population growth rate in any period $t$ from the previous period $t-1$.
    \begin{equation}\label{EqPopDef}
      N_t\equiv\sum_{s=1}^{E+S} \omega_{s,t} \quad\forall t
    \end{equation}
    \begin{equation}\label{EqPopGrowth}
      g_{n,t+1} \equiv \frac{N_{t+1}}{N_t} - 1 \quad\forall t
    \end{equation}
    \begin{equation}\label{EqPopWkDef}
      \tilde{N}_t\equiv\sum_{s=E+1}^{E+S} \omega_{s,t} \quad\forall t
    \end{equation}
    \begin{equation}\label{EqPopWkGrowth}
      \tilde{g}_{n,t+1} \equiv \frac{\tilde{N}_{t+1}}{\tilde{N}_t} - 1 \quad\forall t
    \end{equation}


  \subsection{Households}
    The consumer's maximization problem is:
    \begin{equation}\label{EqUtilMax}
      \begin{split}
        &U_{j,s,t} = \sum_{u=0}^{E+S-s}\beta^u\left[\prod_{v=s-1}^{s+u-1}(1-\rho_v)\right] u\left(c_{j,s+u,t+u},n_{j,s+u,t+u},b_{j,s+u+1,t+u+1}\right) \\
        &\text{where}\quad \rho_{s-1}=0 \\
        &\text{and} \quad u\left(c_{j,s,t},n_{j,s,t},b_{j,s+1,t+1}\right) = \frac{\left(c_{j,s,t}\right)^{1-\sigma} - 1}{1-\sigma} ... \\
        &\qquad\qquad + e^{g_y t(1-\sigma)}\chi^n_s\left(b\left[1 - \left(\frac{n_{j,s,t}}{\tilde{l}}\right)^\upsilon\right]^\frac{1}{\upsilon} + k\right) + \rho_s\chi^b\frac{\left(b_{j,s+1,t+1}\right)^{1-\sigma} - 1}{1-\sigma} \\
        &\text{and} \quad c_{j,s,t} = \prod_{i=1}^I \left( c_{i,j,s,t} - \bar c_{i,s} \right) ^{\alpha_i}; \quad \sum_{i=1}^I \alpha_i = 1 \\
        &\quad\quad\quad\quad\quad\quad\quad\quad\quad\quad\quad\quad\quad\quad\quad\quad\quad\quad\quad\forall j,t\quad\text{and}\:E+1\leq s\leq E+S
      \end{split}
    \end{equation}

    They maximize subject to the following budget constraint.
    \begin{equation}\label{EqBC}
      \begin{split}
        \sum_{i=1}^I p_{i,t}c_{i,j,s,t} + b_{j,s+1,t+1} \leq \left(1 + r_t\right) b_{j,s,t} + w_t e_{j,s}&n_{j,s,t} + \frac{BQ_{j,t}}{\lambda_j\tilde{N}_t} - T_{j,s,t} \\
        \quad\text{where}\quad b_{j,s,1} = 0 \\
        &\text{for} \quad E+1\leq s \leq E+S \quad \forall j,t
      \end{split}
    \end{equation}

    We set up a Lagrangian and solve by taking derivatives with respect to $\{c_{i,j,s,t},n_{j,s,t+u},b_{j,s,t+1}\}$ for all $i,j,s$ and $t$.

    With respect to each consumption good $i$:
    \begin{equation}\label{Eqcfoc}
      \begin{split}
      \frac{\partial U}{\partial c_{i,j,s+u,t+u}} & = \beta^u\left[\prod_{v=s-1}^{s+u-1}(1-\rho_v)\right] \left[ \prod_{i=1}^I \left( c_{i,j,s,t} - \bar c_{i,s} \right) ^{\alpha_i} \right]^{-\sigma}\alpha_i \left( c_{i,j,s,t} - \bar c_{i,s} \right)^{\alpha_i-1} \\
       & - \lambda_{t+u} \left( p_{i,t+u} + \frac{\partial T_{j,s+u,t+u}}{\partial c_{i,j,s+u,t+u}} \right)= 0
        \end{split}
    \end{equation}

    With respect to labor:
    \begin{equation}\label{Eqnfoc}
      \begin{split}
      \frac{\partial U}{\partial n_{j,s+u,t+u}} & = \beta^u\left[\prod_{v=s-1}^{s+u-1}(1-\rho_v)\right] e^{g_y (t+u)(1-\sigma)}\chi^n_{s}\biggl(\frac{b}{\tilde{l}}\biggr)\biggl(\frac{n_{j,s+u,t+u}}{\tilde{l}}\biggr)^{v-1}\Biggl[1 - \biggl(\frac{n_{j,s+u,t+u}}{\tilde{l}}\biggr)\Biggr]^{\frac{1-v}{v}} \\
      & - \lambda_{t+u} \left( w_{+u} e_{j,s+u} - \frac{\partial T_{j,s+u,t+u}}{\partial n_{j,s+u,t+u}} \right)= 0
        \end{split}
    \end{equation}

    With respect to savings:
    \begin{equation}\label{Eqbfoc}
      \begin{split}
      \frac{\partial U}{\partial b_{j,s+u+1,t+u+1}} & = \beta^u\left[\prod_{v=s-1}^{s+u-1}(1-\rho_v)\right] \rho_s\chi^b\bigl(b_{j,s+U+1,t+U+1}\bigr)^{-\sigma} \\
      & - \lambda_{t+u} - \lambda_{t+u+1} \left( 1 + r_{t+u+1} - \frac{\partial T_{j,s+U+1,t+U+1}}{\partial b_{j,s+U+1,t+u+1}} \right)= 0
      \end{split}
    \end{equation}

    We can solve each of these for $\lambda_{t+u}$ to get the following.
    \begin{equation}
      \begin{split}
      \lambda_{t+u} = \frac{ \beta^u\left[\prod_{v=s-1}^{s+u-1}(1-\rho_v)\right] \left[ \prod_{i=1}^I \left( c_{i,j,s,t} - \bar c_{i,s} \right) ^{\alpha_i} \right]^{-\sigma}\alpha_i \left( c_{i,j,s,t} - \bar c_{i,s} \right)^{\alpha_i-1} } { p_{i,t+u} + \frac{\partial T_{j,s+u,t+u}}{\partial c_{i,j,s+u,t+u}} } \nonumber
      \end{split}
    \end{equation}

    \begin{equation}
      \begin{split}
      \lambda_{t+u} = \frac{ \beta^u\left[\prod_{v=s-1}^{s+u-1}(1-\rho_v)\right] e^{g_y (t+u)(1-\sigma)}\chi^n_{s}\biggl(\frac{b}{\tilde{l}}\biggr)\biggl(\frac{n_{j,s+u,t+u}}{\tilde{l}}\biggr)^{v-1}\Biggl[1 - \biggl(\frac{n_{j,s+u,t+u}}{\tilde{l}}\biggr)\Biggr]^{\frac{1-v}{v}} } { w_{+u} e_{j,s+u} - \frac{\partial T_{j,s+u,t+u}}{\partial n_{j,s+u,t+u}} }  \nonumber
      \end{split}
    \end{equation}

    \begin{equation}
      \begin{split}
      \lambda_{t+u} = \beta^u\left[\prod_{v=s-1}^{s+u-1}(1-\rho_v)\right] \rho_s\chi^b\bigl(b_{j,s+U+1,t+U+1}\bigr)^{-\sigma} - \lambda_{t+u+1} \left( 1 + r_{t+u+1} - \frac{\partial T_{j,s+U+1,t+U+1}}{\partial b_{j,s+U+1,t+u+1}} \right)
        \end{split}  \nonumber
    \end{equation}

    These then reduce to the following $I+1$ Euler equations for each $j,s$ and $t$:

    Marginal utility of consumption for each good $i$ compared to the marginal utility of labor:
    \begin{equation}\label{EqcEuler}
      \begin{split}
      & \frac{ \left[ \prod_{i=1}^I \left( c_{i,j,s,t} - \bar c_{i,s} \right) ^{\alpha_i} \right]^{-\sigma}\alpha_i \left( c_{i,j,s,t} - \bar c_{i,s} \right)^{\alpha_i-1} } { p_{i,t+u} + \frac{\partial T_{j,s+u,t+u}}{\partial c_{i,j,s+u,t+u}} } \\
      & = \frac{ e^{g_y (t+u)(1-\sigma)}\chi^n_{s}\biggl(\frac{b}{\tilde{l}}\biggr)\biggl(\frac{n_{j,s+u,t+u}}{\tilde{l}}\biggr)^{v-1}\Biggl[1 - \biggl(\frac{n_{j,s+u,t+u}}{\tilde{l}}\biggr)\Biggr]^{\frac{1-v}{v}} } { w_{t+u} e_{j,s+u} - \frac{\partial T_{j,s+u,t+u}}{\partial n_{j,s+u,t+u}} }
       \end{split}
    \end{equation}

    Intertemporal Euler equation for savings, including the utility effects of bequests:
    \begin{equation}\label{EqbEuler}
      \begin{split}
      & \frac{ e^{g_y (t+u)(1-\sigma)}\chi^n_{s}\biggl(\frac{b}{\tilde{l}}\biggr)\biggl(\frac{n_{j,s+u,t+u}}{\tilde{l}}\biggr)^{v-1}\Biggl[1 - \biggl(\frac{n_{j,s+u,t+u}}{\tilde{l}}\biggr)\Biggr]^{\frac{1-v}{v}} } { w_{t+u} e_{j,s+u} - \frac{\partial T_{j,s+u,t+u}}{\partial n_{j,s+u,t+u}} } = \rho_s\chi^b\bigl(b_{j,s+U+1,t+U+1}\bigr)^{-\sigma}\\ 
      &  - \frac{ \beta(1-\rho_{s+u}) e^{g_y (t+u+1)(1-\sigma)}\chi^n_{s}\biggl(\frac{b}{\tilde{l}}\biggr)\biggl(\frac{n_{j,s+u+1,t+u+1}}{\tilde{l}}\biggr)^{v-1}\Biggl[1 - \biggl(\frac{n_{j,s+u+1,t+u+1}}{\tilde{l}}\biggr)\Biggr]^{\frac{1-v}{v}} } { w_{t+u+1} e_{j,s+u+1} - \frac{\partial T_{j,s+u+1,t+u+1}}{\partial n_{j,s+u+1,t+u+1}} } \times \\
      & \left( 1 + r_{t+u+1} - \frac{\partial T_{j,s+U+1,t+U+1}}{\partial b_{j,s+U+1,t+u+1}} \right)
      \end{split}
    \end{equation}

    % The aggrgeate consumption good is defined as follows.
    % \begin{equation} \label{Eqcagg}
    %     c_{j,s,t}  = \prod_{i=1}^I \left( c_{i,j,s,t} - \bar c_{i,s} \right) ^{\alpha_i} 
    % \end{equation}

    % The price of this aggregate good is given by the following.
    % \begin{equation} \label{Eqpagg}
    %     p_{j,s,t}  = \sum_{i=1}^I p_{i,t} c_{i,j,s,t} 
    % \end{equation}

    An Euler equation that compares marginal utilities of two arbitrary goods ($n$ \& $m$) is given below.
    \begin{equation}\label{EqcEuler2}
      \frac{ \alpha_n \left( c_{n,j,s,t} - \bar c_{n,s} \right)^{\alpha_n-1} } { p_{n,t+u} + \frac{\partial T_{j,s,t}}{\partial c_{n,j,s,t}} } = \frac{ \alpha_m \left( c_{m,j,s,t} - \bar c_{m,s} \right)^{\alpha_m-1} } { p_{m,t+u} + \frac{\partial T_{j,s,t}}{\partial c_{m,j,s,t}} }
    \end{equation}

    We can use this equation for $m \in \{1,2,...,I\}$ solving for $c_{m,j,s,t} - \bar c_{m,s}$.
    \begin{align}
        c_{m,j,s,t} - \bar c_{m,s} & = \left[ \left(c_{n,j,s,t} - \bar c_{n,s} \right)^{1-\alpha_n} \frac{\Gamma_m}{\Gamma_n} \right]^{\frac{1}{1-\alpha_m}}  \label{Eqcmdef} \\
        \Gamma_m & \equiv \frac{ \alpha_m } { p_{m,t} + \frac{\partial T_{j,s,t}}{\partial c_{m,j,s,t}} } \nonumber
    \end{align}

    % We then substitute this into equations \eqref{Eqcagg} and \eqref{Eqpagg} to get:

    % \begin{equation} \label{Eqcagg2}
    %   \begin{split}
    %     c_{j,s,t} & \equiv \prod_{i=1}^I \left[ \left(c_{1,j,s,t} - \bar c_{1,s} \right)^{1-\alpha_1} \frac{\Gamma_i}{\Gamma_1} \right]^{\frac{\alpha_i}{1-\alpha_i}}  \\
    %   \end{split}  
    % \end{equation}

    % \begin{equation} \label{Eqpagg2}
    %     p_{j,s,t}  = \sum_{i=1}^I p_{i,t} \left( \left[ \left(c_{1,j,s,t} - \bar c_{1,s} \right)^{1-\alpha_1} \frac{\Gamma_i}{\Gamma_1} \right]^{\frac{1}{1-\alpha_i}} + \bar c_{i,s} \right) 
    % \end{equation}

    We can solve the household's problem fairly rapidly, if the values of $\frac{\partial T_{j,s,t}}{\partial c_{i,j,s,t}}$ are just constants, as they would be with a typical sales tax.
    \begin{itemize}
    \item First, given $\{p_{i,t}\}_{i=1}^I$ use Euler equation \eqref{EqcEuler} to find the value for $c_{1,j,s,t}$.
    \item Then use equation \eqref{Eqcmdef} to get the rest of the $c_{i,j,s,t}$'s.
    \end{itemize}

  \subsection{Production Goods Firms}
    \subsubsection{Jason's Notes}
      Firms maximize firm value, which is the net present value of dividends less equity issuance:

      \begin{equation}
      \label{eqn:firm_value}
      V_{t} = \sum{u=t}^{\infty} \prod_{\nu=t}{u} \left(\frac{1}{1+r_{\nu}}\right) DIV_{u}-VN_{u},
      \end{equation}

      \noindent\noindent where $DIV_{u}$ are dividend distributions in period $u$ and $VN$ is new equity issuance in period $u$.  The firm's cash flow constraint will give us the value of dividends distributed after investment and earnings (a function of capital and labor) are determined:

      \begin{equation}
      \label{eqn:cash_flow}
      EARN_{u}+VN_{u} = DIV_{u} + I_{u}
      \end{equation}

      \noindent\noindent Here, $I_{u}$ is investment in capital in period $u$ (where we have the price of capital normalized to 1).  Earnings are defined as revenues from the sale of production goods less the price of variable inputs (i.e., labor):

      \begin{equation}
      \label{eqn:earn}
      EARN_{u}= p_{u}F(K_{u},L_{u}) - w_{u}L_{u}
      \end{equation}

      \noindent\noindent Plugging Equation \ref{eqn:earn} and the law of motion for the capital stock into Equation \ref{eqn:cash_flow} yields:

      \begin{equation}
      \label{eqn:cash}
      pF(K_{u},L_{u}) - w_{u}L_{u} + VN_{u} = DIV_{u} + K_{u+1} - (1-\delta)K_{u}
      \end{equation}

      \noindent\noindent We can not find the Belman Equation for the firm's problem by solving for $DIV$ from Equation \ref{eqn:cash2} and substituting the result into Equation \ref{eqn:firm_value}:

      \begin{equation}
      \label{eqn:cash2}
      V(K; r, w) = pF(K,L) - wL  - K' + (1-\delta)K + \frac{1}{1+r}V(K';r',w')
      \end{equation}

      The two FOCs are:

      \begin{equation}
      \label{eqn:foc_k}
      \frac{\partial V(K; r, w)}{\partial K'} : 1= \frac{1}{1+r}\frac{\partial V(K';r',w')}{\partial K'}
      \end{equation}

      \begin{equation}
      \label{eqn:foc_l}
      \frac{\partial V(K; r, w)}{\partial L} : w= \frac{\partial V(K;r,w)}{\partial L}
      \end{equation}

      \noindent\noindent The envelope condition allows us to write \ref{eqn:foc_k} as:

      \begin{equation}
      \label{eqn:foc_k2}
      \frac{\partial V(K; r, w)}{\partial K'} : 1= \frac{1}{1+r}\left[ \frac{\partial F(K',L')}{\partial K'} + 1 -\delta \right]
      \end{equation}

      \subsubsection{Parameterization}

      We will assume that the production function for the firm is a Constant Elasticity of Substitution (CES) function:

      \begin{equation}
      \label{eqn:prod_func}
      F(K,L) = \left(\gamma^{\frac{1}{\varepsilon}}K^{\frac{\varepsilon-1}{\varepsilon}} + (1-\gamma)^{\frac{1}{\varepsilon}}L^{\frac{\varepsilon-1}{\varepsilon}}\right)^{\frac{\varepsilon}{\varepsilon-1}},
      \end{equation}

      \noindent\noindent where $\varepsilon$ is the elasticity of substation between capital and labor and $\gamma$ is the share parameter for the production function (?).  

      Given this parameterization, our two FOCs become:

      \begin{equation}
      \label{eqn:foc_k3}
       r+\delta =\left(\gamma^{\frac{1}{\varepsilon}}K'^{\frac{\varepsilon-1}{\varepsilon}} + (1-\gamma)^{\frac{1}{\varepsilon}}L'^{\frac{\varepsilon-1}{\varepsilon}}\right)^{\frac{1}{\varepsilon-1}}\gamma^{\frac{1}{\varepsilon}}K'^{\frac{-1}{\varepsilon}}
      \end{equation}

      (double check the timing on the interest rate - not sure if it should be the current period or one period ahead - depends upon the timing convention for our notation)

      \begin{equation}
      \label{eqn:foc_l2}
      w= \left(\gamma^{\frac{1}{\varepsilon}}K^{\frac{\varepsilon-1}{\varepsilon}} + (1-\gamma)^{\frac{1}{\varepsilon}}L^{\frac{\varepsilon-1}{\varepsilon}}\right)^{\frac{1}{\varepsilon-1}}(1-\gamma)^{\frac{1}{\varepsilon}}L^{\frac{-1}{\varepsilon}}
      \end{equation}

      I think you should just be able to substitute in these two firm FOCs for the static firm FOCs.  The price out capital and consumption are the same since they are the same good - and we can normalize their price to 1.  Since firms hold capital, we won't have a capital market cleaning condition (I don't think).  But we will have an asset market clearing condition.  This will be that $B_{t}=V_{t}$.  We should be able to solve the infinite geometric series to get the value of $V(\bar{K})$:

      \begin{equation}
      \label{eqn:V_ss}
      \begin{split}
      & V(\bar{K};\bar{r},\bar{w})=F(\bar{K},\bar{L}) - \bar{w}\bar{L}  - \delta\bar{K} + \frac{1}{1+\bar{r}}V(\bar{K};\bar{r},\bar{w})\\
      & \implies V(\bar{K};\bar{r},\bar{w}) = \frac{F(\bar{K},\bar{L}) - \bar{w}\bar{L}  - \delta\bar{K} }{\bar{r}} (1+\bar{r})
      \end{split}
      \end{equation}
       
      To solve for $V_{t}$ outside of the SS, we'll have to use backwards induction.  So, one period before the SS, we have:
       
      \begin{equation}
      \label{eqn:V_Tm1}
      V(K_{T-1};r_{T-1},w_{T-1})=F(K_{T-1},L_{T-1}) - w_{T-1}L_{T-1}  - \bar{K} + (1- \delta)K_{T-1} + \frac{1}{1+\bar{r}}V(\bar{K};\bar{r},\bar{w})
      \end{equation}

      (Again, check the timing for the interest rate).  

      One can then keep iterating backwards from the steady state to the initial period using the recursive relationship as described in Equation \ref{eqn:V_Tm1}.  

    \subsubsection{Kerk's Notes}
      Denote the stock of bonds issued by the firm as $b^F$.  $T^F$ denotes taxes on the firm.

      Firm $\iota$'s maximization problem is:
      \begin{align}
         \begin{split}
         & V_\iota(k^F_{\iota,t},b^F_{\iota,t},\Omega_t) \\
         & = \max_{n^F_{\iota,t},k^F_{\iota,t+1},b^F_{\iota,t+1}} e^F_{\iota,t}  + \frac{1}{1+r_{t+1}} V_\iota(k^F_{\iota,t+1},b^F_{\iota,t+1},\epsilon^F_{\iota,t},\Omega_{t+1}) \label{EqProfMax}
         \end{split} \\
         & e^F_{\iota,t} = p_{\iota,t} y^F_{\iota,t} - w_tn^F_{\iota,t} + (1-\delta)k^F_{\iota,t} - (1+r_t)b^F_{\iota,t} - k^F_{\iota,t+1} + b^F_{\iota,t+1} -T^F_{\iota,t} \label{EqEarn} \\
         & y^F_{\iota,t} = f(z_t,k^F_{\iota,t},n^F_{\iota,t}) = z_t \left[ \zeta (k^F_{\iota,t})^\eta + (1-\zeta)(n^F_{\iota,t})^\eta \right]^{\frac{1}{\eta}} \label{EqProd}
      \end{align}

      The first-order conditions are:
      \begin{align}
         & p_{\iota,t} z_t \left[ \zeta (k^F_{\iota,t})^\eta + (1-\zeta)(n^F_{\iota,t})^\eta \right]^{\frac{1-\eta}{\eta}}(1-\zeta)(n^F_{\iota,t})^{\eta-1} - w_t - \frac{\partial T^F_{\iota,t}}{\partial n^F_{\iota,t}}  = 0 \nonumber\\
         & -1 - \frac{\partial T^F_{\iota,t}}{\partial k^F_{\iota,t+1}} + \frac{1}{1+r_{t+1}} \frac{\partial V_\iota}{\partial k^F_{\iota}}(t+1) = 0 \nonumber \\
         & 1 - \frac{\partial T^F_{\iota,t}}{\partial b^F_{\iota,t+1}} + \frac{1}{1+r_{t+1}} \frac{\partial V_\iota}{\partial b^F_{\iota}}(t+1) = 0 \nonumber 
      \end{align}

      Envelope conditions are:
      \begin{align}
         \frac{\partial V_\iota}{\partial k^F_{\iota}}(t) & = p_{\iota,t} z_t \left[ \zeta (k^F_{\iota,t})^\eta + (1-\zeta)(n^F_{\iota,t})^\eta \right]^{\frac{1-\eta}{\eta}}\zeta (k^F_{\iota,t})^{\eta-1} + 1 - \delta - \frac{\partial T^F_{\iota,t}}{\partial k^F_{\iota,t}}  \nonumber \\
         \frac{\partial V_\iota}{\partial b^F_{\iota}}(t) & = - \left( 1+r_t+\frac{\partial T^F_{\iota,t}}{\partial b^F_{\iota,t}} \right) \nonumber 
      \end{align}

      Euler equations are:
      \begin{align}
         & p_{\iota,t} z_t \left[ \zeta (k^F_{\iota,t})^\eta + (1-\zeta)(n^F_{\iota,t})^\eta \right]^{\frac{1-\eta}{\eta}}(1-\zeta)(n^F_{\iota,t})^{\eta-1} = w_t + \frac{\partial T^F_{\iota,t+1}}{\partial n^F_{\iota,t+1}} \label{EqEulern}\\
         \begin{split}
         & p_{\iota,t+1} z_{t+1} \left[ \zeta (k^F_{\iota,t+1})^\eta + (1-\zeta)(n^F_{\iota,t})^\eta \right]^{\frac{1-\eta}{\eta}}\zeta (k^F_{\iota,t+1})^{\eta-1} \\
         & = \left( 1 + \frac{\partial T^F_{\iota,t+1}}{\partial k^F_{\iota,t+1}} \right) \left( 1+r_{t+1} \right) - 1 - \delta - \frac{\partial T^F_{\iota,t}}{\partial k^F_{\iota,t+1}} \label{EqEulerk} 
         \end{split} \\
         & r_{t+1} = - \frac{ \frac{\partial T^F_{\iota,t+1}}{\partial b^F_{\iota,t+1}} + \frac{\partial T^F_{\iota,t}}{\partial b^F_{\iota,t+1}} } { \frac{\partial T^F_{\iota,t}}{\partial b^F_{\iota,t+1}} } \label{EqEulerb}
      \end{align}

      Starting with a values for $k^F_{\iota,1}$, $b^F_{\iota,1}$, $w_t$, $p_{\iota,t}$ and $z_t$, we get $n^F_{\iota,t}$ from equation \eqref{EqEulern}. Equations \eqref{EqEulerk} and \eqref{EqEulerb} then give $k^F_{\iota,t+1}$ and $b^F_{\iota,t+1}$using the known value of $r_{t+1}$.  This allows us to iteratively solve for labor hired, capital and outstanding debt for firm $\iota$ over time.

      WE NEED TO ADD EQUITY SHARES CHOICE

  \subsection{Consumption Goods Firms}

  \subsection{Government}
    Government will have four functions in our model:
    \begin{enumerate}
    \item The government runs a tax and social security system
      \begin{itemize}
      \item The tax system will be input by the user and/or determined by the current tax law (the default unless the user supplies changes)
      \end{itemize}
    \item The government makes transfers to households outside of the tax/social security system
    \item The government produces output that contributes to private consumption goods (e.g., education)
    \item The government purchases capital and hires labor to produced a non-rival public good (e.g., national defense)
    \end{enumerate}

    \subsubsection{Government budgeting}
      \begin{equation}
      \label{eqn:gbc}
      D_{t+1} + T^{\tau}_{t} = (1+r_{t})D_{t} + T^{H}_{t} + G^{subs}_{t} + G^{emp}_{t} + I^{G}_{t}
      \end{equation}

    \subsubsection{Rule for long-term fiscal stability}
      Let $D_t$ denote the government's outstanding real debt.  $T_t$ is total tax revenue, $T^H_t$ is total household transfers, $G_t$ is government purchases of goods, $L_t$ is the real value of purchases of labor services, and $S_t$ is subsidies to government run firms.

      \begin{equation}
        D_{t+1} = D_t(1+r_t) - T_t + T^H_t + G_t + L_t + S_t
      \end{equation}

      Letting a carat denote the ratio of a variable to GDP, we can rewrite this as follows:

      \begin{equation} \label{EqDhatlom}
        (1+g_{Yt}) \hat D_{t+1} = \hat D_t(1+r_t) - \hat T_t + \hat T^H_t + \hat G_t + \hat L_t + \hat S_t
      \end{equation}

      We need to adopt a government fiscal rule that determines how our residual expenditure $\hat G_t$ evolves over time.

      One way is to adopt a balanced budget rule which keeps the debt-to-GDP ratio constant at it's initial value of $\hat D_0$.

      \begin{align}
        (1+g_{Yt}) \hat D_0 & = \hat D_0(1+r_t) - \hat T_t + \hat T^H_t + \hat G_t + \hat L_t + \hat S_t \nonumber \\
        \hat G_t & = \hat D_0(g_{Yt}-r_t) + \hat T_t - \hat T^H_t -\hat L_t - \hat S_t \label{EqBalBudRule}
      \end{align}

      Another rule is to hold govenrment spending constant and let debt evolve as it will for several period.  Then in period $T$ impose fiscal austerity which forces $\hat G_t$ to adjust over time so that $\hat D_t$ goes to a steady value.

      \begin{equation}
        \hat G_t - \bar G = \rho_t (\hat D_t - \bar D);\quad \rho_t<0 \label{EqAdjRule}
      \end{equation}

      Substituting this into \eqref{EqDhatlom} gives:
      \begin{align}
        (1+g_{Yt}) \hat D_{t+1} & = \hat D_t(1+r_t) - \hat T_t + \hat T^H_t + \rho_t (\hat D_t - \bar D) + \bar G + \hat L_t + \hat S_t \nonumber \\
        \hat D_{t+1} & = \frac{\hat D_t(1+r_t) - \hat T_t + \hat T^H_t + \rho_t (\hat D_t - \bar D) + \bar G + \hat L_t + \hat S_t }{1+g_{Yt}} \label{EqDhatlom2}
      \end{align}

      Consider the steady state version of this.
      \begin{align}
        (1+\bar g_{Y}) \bar D & = \bar D(1+\bar r) + \bar T - \bar T^H + \rho_t (\bar D - \bar D) + \bar G + \bar L + \bar S_t  \nonumber \\
        \bar G & = \bar D(\bar g_{Y} -\bar r) + \bar T - \bar T^H - \bar L - \bar S  \label{EqGbardef}
      \end{align}

      This tells us the long-run value of government spending to GDP that will maintain the debt to GDP target.

      In order for \eqref{EqDhatlom2} to be a contraction mapping over $\hat D$ and thus converge to a steady state, we must put bounds on $\rho_t$.  Rearranging \eqref{EqDhatlom2} and using \eqref{EqGbardef}:

      \begin{align}
        \begin{split}
        (1+g_{Yt}) \hat D_{t+1} & = \hat D_t (1+r_t) - \hat T_t + \hat T^H_t + \rho_t (\hat D_t - \bar D) + \hat L_t + \hat S_t \\
        & + \bar D(\bar g_{Y} - \bar r) + \bar T - \bar T^H - \bar L - \bar S
        \end{split} \nonumber \\
        \begin{split}
        (1+g_{Yt}) \hat D_{t+1} & = \hat D_t (1+r_t) - \hat T_t + \hat T^H_t + \rho_t \hat D_t - \rho_t \bar D + \hat L_t + \hat S_t \\
        & + \bar g_Y \bar D - \bar r \bar D + \bar T - \bar T^H - \bar L - \bar S
        \end{split} \nonumber \\
        \begin{split}
        (1+g_{Yt}) \hat D_{t+1} & =  \hat D_t (1+r_t) + \rho_t (\hat D_t -\bar D) + (\bar g_Y - \bar r ) \bar D \\
        & - (\hat T_t - \bar T) + (\hat T^H_t -\bar T^H) + (\hat L_t -\bar L) + (\hat S_t -\bar S)
        \end{split} \nonumber \\
        \begin{split}
        \hat D_{t+1} - \bar D & = \hat D_t \frac{1+r_t}{1+g_{Yt}} + \frac{\rho_t}{1+g_{Yt}} (\hat D_t -\bar D) + \left( \frac{\bar g_Y - \bar r}{1+g_{Yt}} - 1 \right) \bar D \\
        & + \frac{-(\hat T_t - \bar T) + (\hat T^H_t -\bar T^H) + (\hat L_t -\bar L) + (\hat S_t -\bar S)}{1+g_{Yt}}
        \end{split}  \nonumber \\
        \begin{split} 
        \hat D_{t+1} - \bar D & = \frac{1+r_t+\rho_t}{1+g_{Yt}} (\hat D_t -\bar D) \\
        & + \frac{-(\hat T_t - \bar T) + (\hat T^H_t -\bar T^H) + (\hat L_t -\bar L) + (\hat S_t -\bar S)}{1+g_{Yt}}
        \end{split} 
        \label{EqStab}
      \end{align}

      We need $\frac{\hat D_{t+1} - \bar D}{\hat D_{t} - \bar D} < 1$ for stability.  Equation \eqref{EqStab} gives:
      \begin{align} 
        \frac{\hat D_{t+1} - \bar D}{\hat D_t -\bar D} & = \frac{1+r_t+\rho_t}{1+g_{Yt}}  + \frac{-(\hat T_t - \bar T) + (\hat T^H_t -\bar T^H) + (\hat L_t -\bar L) + (\hat S_t -\bar S)}{(1+g_{Yt})(\hat D_t -\bar D)} < 1 \nonumber \\
        & \frac{1+r_t+\rho_t}{1+g_{Yt}}  < \frac{(\hat T_t - \bar T) - (\hat T^H_t -\bar T^H) - (\hat L_t -\bar L) - (\hat S_t -\bar S)}{(1+g_{Yt})(\hat D_t -\bar D)} \nonumber \\
        & \rho_t  < (1+r_t)\frac{(\hat T_t - \bar T) - (\hat T^H_t -\bar T^H) - (\hat L_t -\bar L) - (\hat S_t -\bar S)}{\hat D_t -\bar D}
      \end{align} 

    \subsubsection{Transfer system}
      We'll need to estimate this.  Probably following \citet{FR1993}.  Or perhaps the micro simulation model calculates some of these.  Or ideally we get something like \citet{KotlikoffXXXX}.

    \subsubsection{Government production of private goods}

    \subsubsection{Government production of public goods}

    \subsubsection{Steps for adding government to the dynamic model}
      \begin{enumerate}
      \item 1 firm
      \item 1 firm + gov't
      \item 2 firms + gov't
      \item tax 2 firms + gov't
      \item N firms with taxes + gov't
      \end{enumerate}

  \subsection{Market Clearing}

  \subsection{Solution and Simulation}

\newpage

\section{Incorporating Feedbacks with Micro Tax Simulations}\label{SecMicro}

  Follow this algorthim:
  \begin{itemize}
    \item Period 1
    \begin{itemize}
      \item Use current IRS public use sample.
      \item Run the following within-period routine
      \begin{itemize}
        \item Do the static tax analysis of this sample, save the results
        \item Summarize the public use sample by aggregating into bins over age and earnings ability
        \item Use this as a starting point for the dynamic macro model
        \item Get values for fundamental interest rates and effective wages for next period
      \end{itemize}
    \end{itemize}
  \item Period 2
    \begin{itemize}
      \item “Age” the public use data demographically by one year.
      \item Let wages and interest rates rise by the amounts predicted in the macro model.
      \item Rerun the within-period routine
    \end{itemize}
  \item Iterate over periods until end of forecast period is reached.
  \end{itemize}

\section{Calibration}
  \subsection{Tax Bend Points}
      We use IRS data which summarizes individual tax returns for 2011 by 19 income categories and 4 filing statuses.  For each filing status we fit the mapping from reported income into adjusted gross income (AGI) using a sufficiently high-order polynomial.  We then use this function to solve for the income level which corresponds to each of the five bend points in the tax code for each filing type.
      \begin{table}[ht]
        \caption{AGI and Income Bend Points}
        \label{Calib_Bend_Tab1}
        \centering
        AGI Bend Points
        \begin{tabular}{|r|r|r|r|r|} \hline 
          Tax rate & Married Joint & Married Separate & Head of Household & Single \\ \hline 
          10\% & 17,400 & 8700 & 12,400 & 8700 \\ \hline 
          15\% & 70,700 & 35,350 & 47,350 & 35,350 \\ \hline 
          25\% & 142,700 & 71,350 & 122,300 & 85,650 \\ \hline 
          28\% & 217,450 & 108,725 & 198,050 & 178,650 \\ \hline 
          33\% & 388,350 & 194,175 & 388,350 & 388,350 \\ \hline 
        \end{tabular}
        \\
        Corresponding Reported Income Bendpoints
        \begin{tabular}{|r|r|r|r|r|} \hline 
          Tax rate & Married Joint & Married Separate & Head of Household & Single \\ \hline 
          0\%  & 5850  & 91 & 756 & 1435 \\ \hline 
          10\% & 22,932 & 8591 & 12,911 & 9956 \\ \hline 
          15\% & 75,181 & 34,592 & 47,023 & 36,021 \\ \hline 
          25\% & 145,866 & 69,768 & 120,200 & 85,244 \\ \hline 
          28\% & 219,162 & 106,245 & 194,176 & 176,270 \\ \hline 
          33\% & 386,798 & 189,674 & 380,043 & 381,524 \\ \hline 
        \end{tabular}
      \end{table}

      We then fit a bivariate probabililty density function over income and filing type from the data.  For each bendpoint we calculate the probability density at that bendpoint and use these as weights in a weighted average over filing types to generate an aggregate bendpoint.
      \begin{table}[ht]
        \caption{Aggregated Bend Points}
        \label{Calib_Bend_Tab2}
        \centering
        \begin{tabular}{|r|r|} \hline 
          Tax rate & Bend Point \\ \hline 
          0\% & 2889 \\ \hline 
          10\% & 15,116 \\ \hline 
          15\% & 52,580 \\ \hline 
          25\% & 114,552 \\ \hline 
          28\% & 196,201 \\ \hline 
          33\% & 380,657 \\ \hline 
        \end{tabular}
      \end{table}

\section{Conclusion}\label{SecConclusion}


\end{spacing}

\newpage
\renewcommand{\theequation}{A.\arabic{section}.\arabic{equation}}
                                                 % redefine the command that creates the section number
\renewcommand{\thesection}{A-\arabic{section}}   % redefine the command that creates the equation number
\setcounter{equation}{0}                         % reset counter
\setcounter{section}{0}                          % reset section number
\section*{APPENDIX}                              % use *-form to suppress numbering



\newpage
\bibliography{AEIBYUDyn}


\end{document}
