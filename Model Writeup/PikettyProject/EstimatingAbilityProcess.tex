	\documentclass[article,11pt,letterpaper,fleqn]{article}
\usepackage{graphicx,color}
\usepackage{array}
\usepackage{threeparttable}
\usepackage[format=hang,font=normalsize,labelfont=bf]{caption}
\usepackage{colortbl}
\usepackage{multirow}
\usepackage{geometry}
\usepackage{subfigure}
\geometry{letterpaper,tmargin=1in,bmargin=1in,lmargin=1.25in,rmargin=1.25in}
\usepackage{hyperref}
\hypersetup{colorlinks,%
citecolor=red,%
filecolor=red,%
linkcolor=red,%
urlcolor=blue,%
pdftex}
\usepackage{amsmath}
\usepackage{amssymb}
\usepackage{amsthm}
\usepackage{harvard}
\usepackage{setspace}
\usepackage{float,graphicx,color}
\usepackage{appendix}
\usepackage{longtable}
\newtheorem*{thm}{Theorem}
\theoremstyle{definition}
\usepackage{lscape}
\numberwithin{equation}{section}
\newcommand{\cn}{\citeasnoun} % shortens command to cite as noun
\newcommand\ve{\varepsilon}


\author{Authors here\thanks{Thanks here.}}
\title{Estimating the ability process}
\date{\today}


% make tables with smaller sized font 
\makeatletter
\def\table{\@ifnextchar[{\table@i}{\table@i[\fps@table]}}
\def\table@i[#1]{\@float{table}[#1]\footnotesize}
\makeatother



%\setlength{\topmargin}{-0.4in}
%\setlength{\topskip}{0.3in}    % between header and text
%\setlength{\textheight}{9.0in} % height of main text
%\setlength{\textwidth}{6in}    % width of text
%\setlength{\oddsidemargin}{39pt} %even side margin
%\setlength{\evensidemargin}{39pt} %odd side margin

\begin{document}
\bibliographystyle{aer}
\maketitle

Let's assume we have data on ability (which we will get by imputing hours onto tax data).  So we have $e_{i,t}$ which is the ability (or log of ability) of household $i$ in year $t$.  We observe some other characteristics of these households, but of these, all that is in our model is age (determined in the tax data by the age of the primary filer).  So let $e_{i,s,t}$ be an observations of household $i$ of age $s$ in year $t$.  We run the OLS regression:

\begin{equation}
\label{eqn:ols}
e_{i,s,t} = \alpha_{i} + \beta_{s} + z_{i,t}
\end{equation}

$\alpha_{i}$ is the household fixed effect.  It gives the mean ability of the household after controlling for age.  $\beta_{s}$ gives mean ability by age. It will give us the average life-cycle profile of ability.  $z_{i,t}$ is the stochastic portion of ability, the residual.

What we do next is to take these $z_{i,t}$ and divide them into centiles.  For each centile, we calculate the mean.  Let $\bar{z}_{p}$ be a vector with the means of $z_{i,t}$ for each centile.  We then calculate the transition matrix for $z_{i,t}$ by finding the fraction that transition between each centile from one year to the next, call this $\pi_{z}$.  

We also divide $\alpha_{i}$ into centiles.  Calc the mean for each centile.  Call this vector of these means $\bar{\alpha}_{p}$.

The model we'd use would have the following states:
\begin{enumerate}
\item model year, $t$
\item age, $s$
\item permanent ability type, $j$
\item stochastic ability, $z$
\item wealth, $b$
\end{enumerate}

We use the data to help use define the state space.  E.g. if we want 100 points the $z$ and $j$ spaces, we just $\bar{z}_{p}$ and $\bar{\alpha}_{p}$ to give us the grid points.  If we want less, we just aggregate up the centiles.  We use the transition matrix $\pi_{z}$ to give the transitions between ability shocks.   In any period, an household's ability is determined by their permanent ability type, their age, and their ability shock.  So that a household of permanent type $j$, age $s$, and with shock $z$ has $e_{j,s,z}=\alpha_{j} + \beta_{s} + z$.  The probability this household has a shock $z'$ tomorrow will depend upon $z$ according the the transition matrix $\pi_{z}$.

\end{document}
