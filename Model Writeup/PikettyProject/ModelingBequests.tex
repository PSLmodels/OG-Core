	\documentclass[article,11pt,letterpaper,fleqn]{article}
\usepackage{graphicx,color}
\usepackage{array}
\usepackage{threeparttable}
\usepackage[format=hang,font=normalsize,labelfont=bf]{caption}
\usepackage{colortbl}
\usepackage{multirow}
\usepackage{geometry}
\usepackage{subfigure}
\geometry{letterpaper,tmargin=1in,bmargin=1in,lmargin=1.25in,rmargin=1.25in}
\usepackage{hyperref}
\hypersetup{colorlinks,%
citecolor=red,%
filecolor=red,%
linkcolor=red,%
urlcolor=blue,%
pdftex}
\usepackage{amsmath}
\usepackage{amssymb}
\usepackage{amsthm}
\usepackage{harvard}
\usepackage{setspace}
\usepackage{float,graphicx,color}
\usepackage{appendix}
\usepackage{longtable}
\newtheorem*{thm}{Theorem}
\theoremstyle{definition}
\usepackage{lscape}
\numberwithin{equation}{section}
\newcommand{\cn}{\citeasnoun} % shortens command to cite as noun
\newcommand\ve{\varepsilon}


\title{Modeling Bequests}
\date{\today}


% make tables with smaller sized font 
\makeatletter
\def\table{\@ifnextchar[{\table@i}{\table@i[\fps@table]}}
\def\table@i[#1]{\@float{table}[#1]\footnotesize}
\makeatother



%\setlength{\topmargin}{-0.4in}
%\setlength{\topskip}{0.3in}    % between header and text
%\setlength{\textheight}{9.0in} % height of main text
%\setlength{\textwidth}{6in}    % width of text
%\setlength{\oddsidemargin}{39pt} %even side margin
%\setlength{\evensidemargin}{39pt} %odd side margin

\begin{document}
\bibliographystyle{aer}
\maketitle

Kerk's formulation of lifetime utility, rewritten:

\begin{equation}
\label{eqn:life_util}
U_{j,s,t} = \sum_{u=0}^{S-s}\beta^{u}\prod_{v=0}^{u}e^{1-p_{j,v+s+E}} u(c_{j,s+u,t+u},n_{j,s+u,t+u}) + \beta^{S-s}\prod_{v=s}^{S}e^{1-p_{j,v+E}} \chi_{b}\left(\frac{(b_{j,S+1,t+S-s+1})^{1-\sigma}-1}{1-\sigma}\right)
\end{equation}

where the per period utility flow is given by:

\begin{equation}
\label{eqn:period_util}
u(c_{j,s,t},n_{j,s,t}) = \frac{(c_{j,s,t}-1)^{1-\sigma}}{1-\sigma} + \chi_{n}e^{g_{y}t(1-\sigma)}\frac{(\tilde{l}-n_{j,s,t})^{1-\eta}}{1-\eta}
\end{equation}

The Lagrangian is thus: 

\begin{equation}
\label{eqn:lagrangian}
\begin{split}
\mathcal{L}_{j,s,t} = &  \sum_{u=0}^{S-s}\left(\beta^{u}\prod_{v=0}^{u}e^{1-p_{j,v+s+E}} u(c_{j,s+u,t+u},n_{j,s+u,t+u})  + \right. \\
 & \left. \lambda_{j,s+u,t+u}\left((1-\tau^{c}_{j,t+u})[(1+r_{t+u})b_{j,s+u,t+u}+w_{t+u}e_{j,s+u}n_{j,s+u,t+u}-b_{j,s+u+1,t+u+1}- \right.\right. \\
 & \left.\left. T^{P}_{j,s+u,t+u} - T^{I}_{j,s+u,t+u}+B_{j,s+u,t+u}]-c_{i,s+u,t+u}\right)\right) + \beta^{S-s}\prod_{v=s}^{S}e^{1-p_{j,v+E}} \chi_{b}\left(\frac{(b_{j,S+1,t+S-s+1})^{1-\sigma}-1}{1-\sigma}\right)
\end{split}	
\end{equation}

Note: I don't know what $T^{I}$ and $T^{P}$ are.  I'm going to assume $T^{P}$ are taxes paid and that they are a function of capital and labor income (separately).  I'll just assume that $T^{I}$ is not a function of income for now.

\ \\

Note: I indexed the consumption tax by year and type, allowing it to vary over time and by ability type (for example, because different groups of people consume different groups of goods - like low income disproportionately consume goods like alcohol and tobacco products which have high excise taxes). 

The FOCs are thus:
\begin{equation}
\label{eqn:foc_c}
\frac{\partial \mathcal{L}_{j,s,t}}{\partial c_{j,s+u,t+u}} \implies c_{j,s+u,t+u}^{-\sigma} = \lambda_{j,s+u,t+u} , \ \ \forall s,t,u
\end{equation}

\begin{equation}
\label{eqn:foc_n}
\begin{split}
\frac{\partial \mathcal{L}_{j,s,t}}{\partial n_{j,s+u,t+u}} \implies &  \chi_{n}e^{g_{y}(t+u)(1-\sigma)}(\tilde{l}-n_{j,s+u,t+u})^{-\eta}= \lambda_{j,s+u,t+u}\left[(1-\tau^{c}_{j,t+u})w_{t+u}e_{j,s+u}\left(1-\frac{\partial T^{P}_{j,s+u,t+u}}{\partial y^{l}}\right)\right]  \\
&  , \ \ \forall s,t,u
\end{split}
\end{equation}

Do I need $e^{g_{y}t(1-\sigma)}$ or $e^{g_{y}(t+u)(1-\sigma)}$ in the above?  I guess the latter should be there, but I don't yet understand this (I get the idea, but haven't worked out the math).

\ \\

Note: $\frac{\partial T^{P}_{j,s+u,t+u}}{\partial y^{l}}$ is the marginal tax rate w.r.t. labor income.

\begin{equation}
\label{eqn:foc_b}
\begin{split}
\frac{\partial \mathcal{L}_{j,s,t}}{\partial b_{j,s+u+1,t+u+1}} \implies &  \lambda_{j,s+u,t+u}(1-\tau^{c}_{j,t+u}) =  
\beta e^{1-p_{j,s+u+1+E}}(1-\tau^{c}_{j,t+u+1}) \\
& \times \left[\lambda_{j,s+u+1,t+u+1}\left(1+\left(1-\frac{\partial T^{P}_{j,s+u+1,t+u+1}}{\partial y^{c}}\right)r_{t+u}\right)\right]   \\
& , \ \ \forall s,t,u \text{ except for } s+u=S
\end{split}
\end{equation}

Note: $\frac{\partial T^{P}_{j,s+u,t+u}}{\partial y^{c}}$ is the marginal tax rate w.r.t. capital income.

\ \\

For bequests we have:

\begin{equation}
\label{eqn:foc_b_final}
\frac{\partial \mathcal{L}_{j,S,t+S-s+1}}{\partial b_{j,S,t+S-s+1}} \implies  \lambda_{j,S,t+S-s}(1-\tau^{c}_{j,t+S-s}) = \chi_{b} (b_{j,S+1,t+S-s})^{-\sigma} 
\end{equation}

Can we calibrate the weight of bequests using consumption/savings data of the very old?  It seems that our model would suggest that $\chi^{b}$ is a function of the ratio of consumption to savings when 80 years old, right?

\end{document}
