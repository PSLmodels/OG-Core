\documentclass[a4paper]{article}

%\usepackage[left=1.25in,top=1in,right=1.25in,nohead,nofoot]{geometry}

\oddsidemargin 0.0in \textwidth 6.5in \topmargin -0.1in \textheight
9.0in


\usepackage{tabularx}

% symbols like \Telefon, \Mobilefone, \Letter and \Email
\usepackage{marvosym}

% package to have clickable website links
%\usepackage[pdftex]{hyperref}
\usepackage{hyperref}
\hypersetup{colorlinks,%
citecolor=red,%
filecolor=red,%
linkcolor=red,%
urlcolor=blue,%
pdftex}

% page numbers
\pagenumbering{arabic}
\usepackage{amsmath}

% multicolumns
\usepackage{palatino, url, multicol}
% setting up links to email
%\renewcommand\UrlLeft{<url: }
%\renewcommand\UrlRight{>}
%\DeclareUrlCommand\email{\urlstyle{rm}%
% \renewcommanad\UrlLeft{e-mail: \  }%
% \renewcommand\UrlRight{}}
\newcommand\email{\begingroup \urlstyle{rm}\Url}


%\usepackage{doublespace}
%\setstretch{1.2}

\usepackage{ae}
\usepackage[T1]{fontenc}
%Define how sections should be laid out
\makeatletter
\renewcommand{\section}{\@startsection{section}{12}
{0mm} % Einzug
{.5\baselineskip} % Vorabstand
{0.2\baselineskip} % Nachabstand
{\large \scshape{\vspace{0.3\baselineskip}}}}
%{\bfseries \scshape \large{\vspace{0.3\baselineskip}}}}
\makeatother

\begin{document}

%\pagestyle{empty}


%Ueberschrift
\begin{center}
\large  \textsc {Simplest model where firms own capital}\\
\small{Supply Side v2.0}\\
\end{center}
\vspace{.5\baselineskip}


Firms maximize firm value, which is the net present value of dividends less equity issuance:

\begin{equation}
\label{eqn:firm_value}
V_{t} = \sum{u=t}^{\infty} \prod_{\nu=t}{u} \left(\frac{1}{1+r_{\nu}}\right) DIV_{u}-VN_{u},
\end{equation}

\noindent\noindent where $DIV_{u}$ are dividend distributions in period $u$ and $VN$ is new equity issuance in period $u$.  The firm's cash flow constraint will give us the value of dividends distributed after investment and earnings (a function of capital and labor) are determined:

\begin{equation}
\label{eqn:cash_flow}
EARN_{u}+VN_{u} = DIV_{u} + I_{u}
\end{equation}

\noindent\noindent Here, $I_{u}$ is investment in capital in period $u$ (where we have the price of capital normalized to 1).  Earnings are defined as revenues from the sale of production goods less the price of variable inputs (i.e., labor):

\begin{equation}
\label{eqn:earn}
EARN_{u}= p_{u}F(K_{u},L_{u}) - w_{u}L_{u}
\end{equation}

\noindent\noindent Plugging Equation \ref{eqn:earn} and the law of motion for the capital stock into Equation \ref{eqn:cash_flow} yields:

\begin{equation}
\label{eqn:cash2}
pF(K_{u},L_{u}) - w_{u}L_{u} + VN_{u} = DIV_{u} + K_{u+1} - (1-\delta)K_{u}
\end{equation}

\noindent\noindent We can not find the Belman Equation for the firm's problem by solving for $DIV$ from Equation \ref{eqn:cash2} and substituting the result into Equation \ref{eqn:firm_value}:

\begin{equation}
\label{eqn:cash2}
V(K; r, w) = pF(K,L) - wL  - K' + (1-\delta)K + \frac{1}{1+r}V(K';r',w')
\end{equation}

The two FOCs are:

\begin{equation}
\label{eqn:foc_k}
\frac{\partial V(K; r, w)}{\partial K'} : 1= \frac{1}{1+r}\frac{\partial V(K';r',w')}{\partial K'}
\end{equation}

\begin{equation}
\label{eqn:foc_l}
\frac{\partial V(K; r, w)}{\partial L} : w= \frac{\partial V(K;r,w)}{\partial L}
\end{equation}

\noindent\noindent The envelope condition allows us to write \ref{eqn:foc_k} as:

\begin{equation}
\label{eqn:foc_k}
\frac{\partial V(K; r, w)}{\partial K'} : 1= \frac{1}{1+r}\left[ \frac{\partial F(K',L')}{\partial K'} + 1 -\delta \right]
\end{equation}

\section{Parameterization}

We will assume that the production function for the firm is a Constant Elasticity of Substitution (CES) function:

\begin{equation}
\label{eqn:prod_func}
F(K,L) = \left(\gamma^{\frac{1}{\varepsilon}}K^{\frac{\varepsilon-1}{\varepsilon}} + (1-\gamma)^{\frac{1}{\varepsilon}}L^{\frac{\varepsilon-1}{\varepsilon}}\right)^{\frac{\varepsilon}{\varepsilon-1}},
\end{equation}

\noindent\noindent where $\varepsilon$ is the elasticity of substation between capital and labor and $\gamma$ is the share parameter for the production function (?).  

Given this parameterization, our two FOCs become:

\begin{equation}
\label{eqn:foc_k}
 r+\delta =\left(\gamma^{\frac{1}{\varepsilon}}K'^{\frac{\varepsilon-1}{\varepsilon}} + (1-\gamma)^{\frac{1}{\varepsilon}}L'^{\frac{\varepsilon-1}{\varepsilon}}\right)^{\frac{1}{\varepsilon-1}}\gamma^{\frac{1}{\varepsilon}}K'^{\frac{-1}{\varepsilon}}
\end{equation}

(double check the timing on the interest rate - not sure if it should be the current period or one period ahead - depends upon the timing convention for our notation)

\begin{equation}
\label{eqn:foc_l}
w= \left(\gamma^{\frac{1}{\varepsilon}}K^{\frac{\varepsilon-1}{\varepsilon}} + (1-\gamma)^{\frac{1}{\varepsilon}}L^{\frac{\varepsilon-1}{\varepsilon}}\right)^{\frac{1}{\varepsilon-1}}(1-\gamma)^{\frac{1}{\varepsilon}}L^{\frac{-1}{\varepsilon}}
\end{equation}

I think you should just be able to substitute in these two firm FOCs for the static firm FOCs.  The price out capital and consumption are the same since they are the same good - and we can normalize their price to 1.  Since firms hold capital, we won't have a capital market cleaning condition (I don't think).  But we will have an asset market clearing condition.  This will be that $B_{t}=V_{t}$.  We should be able to solve the infinite geometric series to get the value of $V(\bar{K})$:

\begin{equation}
\label{eqn:V_ss}
\begin{split}
& V(\bar{K};\bar{r},\bar{w})=F(\bar{K},\bar{L}) - \bar{w}\bar{L}  - \delta\bar{K} + \frac{1}{1+\bar{r}}V(\bar{K};\bar{r},\bar{w})\\
& \implies V(\bar{K};\bar{r},\bar{w}) = \frac{F(\bar{K},\bar{L}) - \bar{w}\bar{L}  - \delta\bar{K} }{\bar{r}} (1+\bar{r})
\end{split}
\end{equation}
 
 To solve for $V_{t}$ outside of the SS, we'll have to use backwards induction.  So, one period before the SS, we have:
 
\begin{equation}
\label{eqn:V_Tm1}
V(K_{T-1};r_{T-1},w_{T-1})=F(K_{T-1},L_{T-1}) - w_{T-1}L_{T-1}  - \bar{K} + (1- \delta)K_{T-1} + \frac{1}{1+\bar{r}}V(\bar{K};\bar{r},\bar{w})
\end{equation}
\end{document}

(Again, check the timing for the interest rate).  

One can then keep iterating backwards from the steady state to the initial period using the recursive relationship as described in Equation \ref{eqn:V_Tm1}.  

\end{document}