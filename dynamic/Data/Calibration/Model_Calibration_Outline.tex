	\documentclass[article,11pt,letterpaper,fleqn]{article}
\usepackage{graphicx,color}
\usepackage{array}
\usepackage{threeparttable}
\usepackage[format=hang,font=normalsize,labelfont=bf]{caption}
\usepackage{colortbl}
\usepackage{multirow}
\usepackage{geometry}
\usepackage{subfigure}
\geometry{letterpaper,tmargin=1in,bmargin=1in,lmargin=1.25in,rmargin=1.25in}
\usepackage{hyperref}
\hypersetup{colorlinks,%
citecolor=red,%
filecolor=red,%
linkcolor=red,%
urlcolor=blue,%
pdftex}
\usepackage{amsmath}
\usepackage{amssymb}
\usepackage{amsthm}
\usepackage{harvard}
\usepackage{setspace}
\usepackage{float,graphicx,color}
\usepackage{appendix}
\usepackage{longtable}
\newtheorem*{thm}{Theorem}
\theoremstyle{definition}
\usepackage{lscape}
\numberwithin{equation}{section}
\newcommand{\cn}{\citeasnoun} % shortens command to cite as noun
\newcommand\ve{\varepsilon}


\author{Authors here\thanks{Thanks here.}}
\title{Calibration of the OLG Dynamic Scoring Model}
\date{\today}


% make tables with smaller sized font 
\makeatletter
\def\table{\@ifnextchar[{\table@i}{\table@i[\fps@table]}}
\def\table@i[#1]{\@float{table}[#1]\footnotesize}
\makeatother



%\setlength{\topmargin}{-0.4in}
%\setlength{\topskip}{0.3in}    % between header and text
%\setlength{\textheight}{9.0in} % height of main text
%\setlength{\textwidth}{6in}    % width of text
%\setlength{\oddsidemargin}{39pt} %even side margin
%\setlength{\evensidemargin}{39pt} %odd side margin

\begin{document}
\bibliographystyle{aer}
\maketitle



\begin{abstract}
This note outlines the parameters of the model, how they will be calibrated, and the data used to do so.
\end{abstract}

\section{Parameters of the model}
\label{sec:params}

Tables \ref{tab:params_demand} and \ref{tab:params_supply} outline the parameters of the demand side and supply sides of the model, respectively.

% Table generated by Excel2LaTeX from sheet 'DemandSideVariableList'
\begin{table}[htbp]
  \centering
  \caption{Model Parameters, Demand Side}
    \begin{tabular}{ll}
    \hline
    \hline
    Parameter & Description \\
    \hline
    Consumer Preferences &  \\
    \ \ \ $\sigma$ & Coefficient of relative risk aversion \\
    \ \ \ $\eta$ & Frisch elasticity of labor supply \\
    \ \ \ $\chi_{n,s}$ & Age specific utility weight for labor supply \\
    \ \ \ $\chi_{b}$ & Utility weight of bequest in year $S+E$ \\
    \ \ \ $\beta$ & Rate of time preference \\
    Population Dynamics &  \\
    \ \ \ $E$ & Age enter economic life \\
    \ \ \ $S$ & Maximum number of years of economic life \\
    \ \ \ $f_{s}$ & Age specific fertility rate \\
    \ \ \ $i_{s}$ & Age specific net immigration rate \\
    \ \ \ $\rho_{s}$ & Age specific mortality rate \\
    \ \ \ $N_{0}$ & Initial population \\
    \ \ \ $w_{s,0}$ & Initial distribution of ages \\
    Labor Supply/Ability &  \\
    \ \ \ $J$ & Number of permanent ability types \\
    \ \ \ $\lambda_{j}$ & Fraction born into each ability type \\
    \ \ \ $\tilde{l}$ & Labor time endowment \\
    Composite Consumption &  \\
    \ \ \ $I$ & Number of consumption goods \\
    \ \ \ $b_{i,s}$ & Minimum required consumption of good $i$ at age $s$ \\
    \ \ \ $\beta_{i,s}$ & Stone-Geary share parameter of good $i$ at age $s$ \\
    \ \ \ $Z$ & ``Transition" matrix relating outputs from firms to consumption goods \\
     & (dimesions are $I\times M$) \\
    \ \ \ $\gamma_{m}$ & Preference weight for corporate good from industry $m$ in CES function  \\
    \ \ \ $\varepsilon_{C}$ & Elasticity of substitution between corporate and non-corporate outputs \\
    &  in CES function. \\
    \hline
    \hline
    \end{tabular}%
  \label{tab:params_demand}%
\end{table}%


% Table generated by Excel2LaTeX from sheet 'SupplySideVariableList'
\begin{table}[htbp]
  \centering
  \caption{Model Parameters, Supply Side}
    \begin{tabular}{ll}
    \hline
    \hline
    Parameter & Description \\
    \hline
    Production Function &  \\
    \ \ \ $\gamma^{C}_{m}$ & Capital weighting in CES production function \\
    \ \ \ $\epsilon^{C}_{m}$ & Elasticity of substitution of capital for labor in CES production function \\
    \ \ \ $\delta^{C}_{m}$ & Rate of economic depreciation on capital stock in the corporate sector \\
    \ \ \ $\beta^{C}_{m}$ & Scaling parameter for quadratic investment adjustment costs \\
    \ \ \ $\mu^{C}_{m}$ & Steady-state investment rate \\
    Economic Growth &  \\
    \ \ \ $n$ & Rate of population growth (exogenous and implied by fertility, mortality, and net immigration rates) \\
    \ \ \ $g_{y}$ & Rate of labor augmenting productivity growth (exogenous) \\
    Financial Policy &  \\
    \ \ \ $\zeta^{C}_{m}$ & Fraction of earnings paid out in dividends \\
    \ \ \ $b^{C}_{m}$ & Debt/Capital ratio \\
    Tax Policy &  \\
    \ \ \ $\tau^{b}_{s}$ & Corporate business income tax rate \\
    \ \ \ $\delta^{\tau C}_{m,s}$ & Rate of tax depreciation on corporate capital \\
    \ \ \ $\tau^{pC}_{s}$ & Property tax rate on corporate capital \\
    \ \ \ $\tau^{i}_{s}$ & Individual income tax rate on interest income \\
    \ \ \ $\tau^{g}_{s}$ & Individual income tax rate on capital gains \\
    \ \ \ $f_{e}$ & Dummy variable for full expensing of investment  \\
    \ \ \ $f_{i}$ & Dummy variable for deductibility of corporate interest paid \\
    \ \ \ $f_{p}$ & Dummy variable for deductibility of repayment of principle on loans \\
    \ \ \ $f_{b}$ & Dummy variable for inclusion of proceeds of loan in corp income tax base \\
    \ \ \ $f_{d}$ & Dummy variable for deductibility of depreciation expenses \\
    Population of firms &  \\
    \ \ \ $M$ & Number of industries \\
    \ \ \ $\Omega$ & ``Transition" matrix relating output os firms to the supply of new capital \\
    & (dimensions are $M\times M$) \\
    Multi-nationals &  \\
    \ \ \ TBD & Will need to parameterize the share of MNCs, the amount of profits overseas, the elasticity of \\
    & profit shifting w.r.t. tax rates, foreign tax rates, and more \\
    \hline
    \hline
    \end{tabular}%
  \label{tab:params_supply}%
\end{table}%

In addition, there are other parameters of fiscal and monetary policy that need to be modeled (outside of the tax rates noted in the tables above).  Table \ref{params_govt} summarizes these.

% Table generated by Excel2LaTeX from sheet 'GovtVariableList'
\begin{table}[htbp]
  \centering
  \caption{Model Parameters, Fiscal and Monetary Policies}
    \begin{tabular}{lll}
    \hline
    \hline
    Parameter & Description & Calibration \\
    \hline
    \ \ \ $\mu$ & Gov't debt to GDP response rate & Estimate gov't response function using  \\
    &Parametrizes gov't adjustments to spending to &data on debt to gdp ratio and gdp \\
    &  return to constant debt to GDP ratio in steady state.  & \\
    \ \ \ $\bar{d}$ & User supplied parameter for gov't debt to GDP ratio in SS & User set \\
    &  (may also be model implied based on tax policy chosen) & \\
    \ \ \ $T(\cdot,\cdot,\cdot)$ & Individual net tax function.   & Micro-sim model \\
    & A function of labor and capital income, perhaps age. &  \\
    \ \ \ Monetary Policy & Fed reaction function & Unclear - likely do several cases. \\
    \hline
    \hline
    \end{tabular}%
  \label{tab:params_govt}%
\end{table}%

Finally, we have to choose parameters relating to the model solution algorithm.  These are summarized in Table \ref{tab:params_TPI}.

% Table generated by Excel2LaTeX from sheet 'ModelSolutionParameters'
\begin{table}[htbp]
  \centering
  \caption{Model Parameters, Solution Algorithm}
    \begin{tabular}{lll}
    \hline
    \hline
    Parameter & Description & Value \\
    \hline
    \ \ \ $T$ & Number of periods until economy converges to steady state & 120 \\
    \ \ \ $\nu$ & Dampening parameter for TPI & 0.2 \\
    \hline
    \hline
    \end{tabular}%
  \label{tab:params_TPI}%
\end{table}%


\section{Calibrating the Demand Side}
\label{sec:demand_calib}

Equations and data used to calibrate the demand sides are summarized in Table \ref{tab:calib_demand}.

% Table generated by Excel2LaTeX from sheet 'DemandSideCalibration'
\begin{landscape}
\begin{table}[htbp]
  \centering
  \caption{Calibration, Demand Side of Model}
    \begin{tabular}{lll}
    \hline
    \hline
    Parameter & Calibration & Data \\
    \hline
    Consumer Preferences &       &  \\
    \ \ \ $\sigma$ & Set using standard values from lit (e.g., $\sigma=1.5$) &  \\
    \ \ \ $\eta$ & Set using standard values from lit (e.g., $\eta=2$? &  \\
    \ \ \ $\chi_{n,s}$ &       & Wages and hours by age?  CPS? \\
    \ \ \ $\chi_{b}$ & $\chi_{b}=\frac{c_{E+s}}{bq_{E+S}}$ & Data avg consumption and bequest in last year of life (use ages 80-100?) \\
    \ \ \ $\beta$ & $\frac{1}{\beta}=1+(1-\tau_{i})\bar{r}$ & Data on after tax real interest rates \\
    Population Dynamics &       &  \\
    \ \ \ $E$ & Set to 20 &  \\
    \ \ \ $S$ & Set to 80 &  \\
    \ \ \ $f_{s}$ & Fraction having children by age & US vital stats \\
    \ \ \ $i_{s}$ & Residual from pop change given $f_{s}$ and $\rho_{s}$ &  \\
    \ \ \ $\rho_{s}$ & Fraction dying by age & US vital stats \\
    \ \ \ $N_{0}$ & Set based on population in base year &  \\
    \ \ \ $w_{s,0}$ & Set based on population in base year &  \\
    Labor Supply/Ability &       &  \\
    \ \ \ $J$ & Set to 12 &  \\
    \ \ \ $\lambda_{j}$ & So that steady state distribution &  \\
    \ \ \ $\tilde{l}$ & Set to some max usable hours in a year (e.g. 4000) &  \\
    Composite Consumption &       &  \\
    \ \ \ $I$ & Set to 17 &  \\
    \ \ \ $b_{i,s}$ & See Fullerton and Rogers (1983) & Consumption data by age and consumption good category. \\
    \ \ \ $\beta_{i,s}$ & See Fullerton and Rogers (1983) & Consumption data by age and consumption good category. \\
    \ \ \ $Z$ & See Fullerton and Rogers (1983) &  \\
    \ \ \ $\gamma_{m}$ & Fraction of corp output by industry & Data on output by corp/non-corp and industry \\
    \ \ \ $\varepsilon_{C}$ & See Fullerton and Rogers (1983) & Data on output by corp/non-corp and industry \\
    \hline
    \hline
    \end{tabular}%
  \label{tab:calib_demand}%
\end{table}%
\end{landscape}

The composite consumption good for each individual will be made up goods from the $I$ consumption categories according to a Stone-Geary utility function.  This function will result in individuals with different incomes and ages consuming different fractions of the consumption goods.  Thus, we are better able to model the incidence of consumption taxes and income taxes that differentially affect the prices of consumer goods.  Table \ref{tab:cons_goods} summarizes the consumption good categories we consider.

% Table generated by Excel2LaTeX from sheet 'ConsumptionGoodsCategories'
\begin{table}[htbp]
  \centering
  \caption{Consumption Goods Categories}
    \begin{tabular}{lll}
    \hline
    \hline
   \# & Consumption Good Category \\
    \hline
    1     & Food  \\
    2     & Alcohol \\
    3     & Tobacco \\
    4     & Household fuels and utilities \\
    5     & Shelter \\
    6     & Furnishings \\
    7     & Appliances \\
    8     & Apparel \\
    9     & Public transportation \\
    10    & New and used cars, fees, and maintenance \\
    11    & Cash contributions and personal care (personal services) \\
    12    & Financial services \\
    13    & Reading and entertainment (recreation) \\
    14    & Household operations (nondurables) \\
    15    & Gasoline and motor oil \\
    16    & Health care \\
    17    & Education \\
     \hline
    \hline
    \end{tabular}%
  \label{tab:cons_goods}%
\end{table}%

Note that these categories do not map directly into the production sectors described in Section \ref{sec:supply_calib}.  To map production goods into consumption goods we use a fixed coefficient model summarized by the ``transition matrix" $Z$.  Furthermore, individuals have preferences over the consumption of the same types of goods from both the corporate and non-corporate sectors.  Consumer preferences across sectors are summarized by a Constant Elasticity of Substitution (CES) utility function, the parameters of which we estimate.

The calibration of the parameters of the composite consumption good is outlined in \cn{FR1993}.  The process of estimating the parameters of the the CES function for preferences over corporate and non-corporate goods and for estimating the parameters of the Stone-Geary function describing preferences over goods from different consumption categories should be directly analogous (with differences being the number of industries considered and the vintage of the data used).

\section{Supply Side Calibration}
\label{sec:supply_calib}

Equations and data used to calibrate the demand sides are summarized in Table \ref{tab:calib_demand}.  Many of the parameters must be found for each industry, $m$, and sector $C\in\{\text{Corporate, Noncorporate}\}$.

% Table generated by Excel2LaTeX from sheet 'SupplySideVariableCalibration'
\begin{landscape}
\begin{table}[htbp]
  \centering
  \caption{Calibration, Supply Side of Model}
    \begin{tabular}{lll}
    \hline
    \hline
    Parameter & Calibration & Data \\
    \hline
    Production Function &       &  \\
    \ \ \ $\gamma^{C}_{m}$ & Need capital and labor shares by industry/sector & BEA national accounts data? \\
    \ \ \ $\epsilon^{C}_{m}$ & ??    &  \\
    \ \ \ $\delta^{C}_{m}$ & Weight avg of depreciation rates by capital type,  & BEA estimated deprec rates, BEA capital stock data \\
    & amount of capital by type by industry/sector & \\
    \ \ \ $\beta^{C}_{m}$ & ??    &  \\
    \ \ \ $\mu^{C}_{m}$ & A function of $g_{y}$, $n$, and $\delta^{C}_{m}$ &  \\
    Economic Growth &       &  \\
    \ \ \ $n$ & Implied by $f_{s}$, $i_{s}$, and $\rho_{s}$ &  \\
    \ \ \ $g_{y}$ & Avg GDP growth rate? &  \\
    Financial Policy &       &  \\
    \ \ \ $\zeta^{C}_{m}$ & Dividend payout ratio by industry &  \\
    \ \ \ $b^{C}_{m}$ & $b^{C}_{m}=\frac{B^{C}_{m}}{K^{C}_{m}}$ & Total debt and capital by industry, Flow of Funds and BEA cap stock \\
    Tax Policy &       &  \\
    \ \ \ $\tau^{b}_{s}$ & Top statutory corp rate,  &  \\
    & Avg marginal rate for owners of non-corp entities (from micro sim model?)& \\
    \ \ \ $\delta^{\tau C}_{m,s}$ & Weighted avg of tax deprecation rates by capital type & Amount of capital by type by industry/sector, mapped to tax deprecation rates by capital type \\
    \ \ \ $\tau^{pC}_{s}$ & User set?? &  \\
    \ \ \ $\tau^{i}_{s}$ & Use marginal tax rate of the median individual with interest income &  \\
    \ \ \ $\tau^{g}_{s}$ & Use marginal tax rate of the median individual with capital gains income &  \\
    \ \ \ $f_{e}$ & User set &  \\
    \ \ \ $f_{i}$ & User set &  \\
    \ \ \ $f_{p}$ & User set &  \\
    \ \ \ $f_{b}$ & User set &  \\
    \ \ \ $f_{d}$ & User set &  \\
    Population of firms &       &  \\
    $M$   & Set to 24 & NAICS Classification Codes \\
    $\Omega$ &       & BEA Input-Output Tables \\
     \hline
    \hline
    \end{tabular}%
  \label{tab:calib_supply}%
\end{table}%
\end{landscape}

Note that the exogenous dividend payout ratio, $\zeta^{C}_{m}$, will only apply to the corporate sector.  Non-corporate entities may not retain earnings so payout all earnings not immediately reinvested.

``Capital type" is not really clear.  Probably the best thing to do is to look at the BEA classification of asset types (which is very detailed) and try to map that to the various asset lives used for tax policy.  This should be detailed enough for use when calculating economic depreciation and will be very useful when calculating the tax deprecation rates by sector and industry.  There are two options to make this mapping: 1) Find the economic life of the asset as determined by the BEA and then assign that a tax type that has a similar (but shorter, since tax depreciation is accelerated) depreciable life. or 2) Use the asset type descriptions from BEA and then map that into the assets descriptions for the various asset types used in tax (see the Form 4562 instructions: \url{http://www.irs.gov/pub/irs-pdf/i4562.pdf}).  We also have to be careful that we don't just consider assets of the type of Form 4562 (\url{http://www.irs.gov/pub/irs-pdf/f4562.pdf}), but also consider assets that received immediate expensing for tax purposes (like intangibles) and inventories.

To get the economic deprecation rate by industry and sector, we'll take a weighted average.  Assume there are $I$ types of capital.  We use the depreciation rate for each of those $I$ types find the weighted average where the weights are determined by the amount of capital of each type.  Thus the economic depreciation rate for capital in sector $C$ in industry $m$ can be give by:

\begin{equation}
\label{eqn:econ_deprec}
\delta^{C}_{m}=\sum_{i=1}{I}\delta_{i}\frac{K^{C}_{i,m}}{K^{C}_{m}},
\end{equation}

\noindent\noindent where $K^{C}_{i,m}$ is the amount of capital of type $i$ in sector $C$ in industry $m$ and $K^{C}_{m}$ is the total amount of capital in sector $C$ in industry $m$.  Economics depreciation rates, $\delta_{i}$ will be found through the BEA's estimated depreciation rates by asset type.  The tax rate of depreciations will be calculated analogously: 

\begin{equation}
\label{eqn:tax_deprec}
\delta^{\tau C}_{m}=\sum_{i=1}{I}\delta^{\tau C}_{i}\frac{K^{C}_{i,m}}{K^{C}_{m}},
\end{equation}

\noindent\noindent where $\delta^{\tau C}_{i}$ is the tax depreciation rate  of capital of type $i$ and is given by tax law.

Table \ref{tab:prod_ind} summarizes the production industries we consider.\footnote{This excludes the multi-national sector, which we still need to think about.}

% Table generated by Excel2LaTeX from sheet 'ProductionIndustries'
\begin{table}[htbp]
  \centering
  \caption{Production Industries}
    \begin{tabular}{lll}
    \hline
    \hline
    \# & NAICS Code & Industry \\
    \hline
    1     & 11    & Agriculture, Forestry, Fishing and Hunting \\
    2     & 211   & Oil and Gas Extraction \\
    3     & 212 and 213 & Mining and Support Activities for Mining \\
    4     & 22    & Utilities \\
    5     & 23    & Construction \\
    6     & 32411 & Petroleum Refineries \\
    7     & 336   & Transportation Equipment Manufacturing \\
    8     & 3391  & Medical Equipment and Supplies Manufacturing \\
    9     & Other codes in 31-33 & Manufacturing \\
    10    & 42    & Wholesale Trade \\
    11    & 44-45 & Retail Trade \\
    12    & 48-49 & Transportation and Warehousing \\
    13    & 51    & Information \\
    14    & 52    & Finance and Insurance \\
    15    & 53    & Real Estate and Rental and Leasing \\
    16    & 54    & Professional, Scientific, and Technical Services \\
    17    & 55    & Management of Companies and Enterprises \\
    18    & 56    & Administrative and Support and Waste Management and Remediation Services \\
    19    & 61    & Educational Services \\
    20    & 62    & Health Care and Social Assistance \\
    21    & 71    & Arts, Entertainment, and Recreation \\
    22    & 72    & Accommodation and Food Services \\
    23    & 81    & Other Services (except Public Administration) \\
    24    & 92    & Public Administration \\
      \hline
    \hline
    \end{tabular}%
  \label{tab:prod_ind}%
\end{table}%


section{Summary of data sources}
\label{sec:data}

These are mostly just for the data we do not yet have together/parameters not yet calibrated. 

\begin{itemize}
\item Consumption: Consumer Expenditure Survey (CEX) (for as many years as possible)
	\begin{itemize}
	\item Consumption by category and age
	\item Use categories in Table \ref{tab:cons_goods} or finer categories that can be aggregated up
	\end{itemize}
\item Production: BEA: National Account Data (?) (for as many years as possible)
	\begin{itemize}
	\item Output by sector (corporate and non-corporate) and industry
	\item Capital by sector and industry and capital type
	\item Investment by sector and industry
	\item Income shares by sector and industry
		\begin{itemize}
		\item Payments to capital (Dividends and interest income) (may need Flow of Funds data)
		\item Payments to labor (wages, salaries, benefits) (may need Flow of Funds data)
		\end{itemize}
	\item Use industries as in Table \ref{tab:prod_ind} or finer industries groupings that can be aggregated up
	\end{itemize}
\item Economic Depreciation: BEA Estimated Depreciation Rates (\url{http://www.bea.gov/national/pdf/fixed\%20assets/BEA_depreciation_2013.pdf})	
\item Mapping production output from each sector to capital type by sector: BEA Input-Output Tables (?) (\url{http://www.bea.gov/industry/io_annual.htm})
	\begin{itemize}
	\item Latest tables are probably fine.
	\end{itemize}
\item Firm financial policy: Flow of Funds Data (for as many years as possible)
	\begin{itemize}
	\item Earnings by sector and industry 
	\item Dividends paid by corp sector by industry
	\item Debt held by sector and industry
	\end{itemize}
\item Disutility of labor: Current Population Survey (CPS) March Supplement (for as many years as possible)
	\begin{itemize}
	\item Hours and wages by age
	\item Note: This is already calibrated, but we might want to think about the parameter further
	\end{itemize}
\item Mapping output of each production sector to categories of consumption goods: Survey of Current Business, ``Make and Use Tables"
	\begin{itemize}
	\item Latest tables are probably fine.
	\end{itemize}
\item Bequests: Data???
	\begin{itemize}
	\item Not sure of data.
	\item Survey of Consumer Finances?  Estate Tax Returns?  Other?
	\item Might want to do a lit review here.  
	\end{itemize}
\end{itemize}

\section{Useful references}
\label{sec:refs}

\cn{FR1993} outline in great detail the calibration of the composite consumption preference parameters.  They also outline a process for determining the transition matrix that can be used to map production output to the new capital available for investment.

\cn{Temple2012} and \cn{KMW2012} describe the calibration of CES functions generally.


\bibliography{TaxModelCalibrationReferences}

\end{document}
