\documentclass[article,11pt,letterpaper,fleqn]{article}
\usepackage{graphicx,color}
\usepackage{array}
\usepackage{threeparttable}
\usepackage[format=hang,font=normalsize,labelfont=bf]{caption}
\usepackage{colortbl}
\usepackage{multirow}
\usepackage{geometry}
\usepackage{subfigure}
\geometry{letterpaper,tmargin=1in,bmargin=1in,lmargin=1.25in,rmargin=1.25in}
\usepackage{hyperref}
\hypersetup{colorlinks,%
citecolor=red,%
filecolor=red,%
linkcolor=red,%
urlcolor=blue,%
pdftex}
\usepackage{amsmath}
\usepackage{amssymb}
\usepackage{amsthm}
\usepackage{harvard}
\usepackage{tikz}
\usepackage{setspace}
\usepackage{float,graphicx,color}
\usepackage{appendix}
\usepackage{longtable}
\newtheorem*{thm}{Theorem}
\theoremstyle{definition}
\usepackage{lscape}
\numberwithin{equation}{section}
\newcommand{\cn}{\citeasnoun} % shortens command to cite as noun
\newcommand\ve{\varepsilon}


\title{Guide to Calibration of Household Income Processes in the OLG Dynamic Scoring Model}
\date{\today}



% make tables with smaller sized font 
\makeatletter
\def\table{\@ifnextchar[{\table@i}{\table@i[\fps@table]}}
\def\table@i[#1]{\@float{table}[#1]\footnotesize}
\makeatother



%\setlength{\topmargin}{-0.4in}
%\setlength{\topskip}{0.3in}    % between header and text
%\setlength{\textheight}{9.0in} % height of main text
%\setlength{\textwidth}{6in}    % width of text
%\setlength{\oddsidemargin}{39pt} %even side margin
%\setlength{\evensidemargin}{39pt} %odd side margin

\begin{document}
\bibliographystyle{aer}
\maketitle



\begin{abstract}
This will be the section in the dynamic scoring model handbook on calibrating parameters of household income processes.
\end{abstract}

\section{Panel Study of Income Dynamics}

To calibrate the parameters defining household income processes, we use the Panel Study of Income Dynamics (PSID).  We use the years 19XX-20XX.  From these, we draw the following variables...  All dollar values are deflated to 20XX dollars/ 

\subsection{Sample Selection}

We create a unique household identifier for combinations of head and spouse.  That is, we follow the household unit and not the individual.  We define the age of a household by the age of the head.

\section{Lifetime income}

In our model, lifetime income is endogenous.  We therefore seek to estimate the process for effective labor units for households over their lifecycle.  We define households by their lifetime income group.  Again, since income is endogenous, we create lifetime income groups by considering the expected, discounted expected value of lifetime labor endowments.\footnote{Note that the process for effective labor units is deterministic.  The uncertainty is over the exogenous probability of mortality.  The probability of mortality at each age, $s$, is the same across income groups, but can still affect the determination of different lifetime income groups since the probability of mortality varies by age, thus interacting with the lifecycle profile of labor endowments.} Our measure of lifetime income is the value of lifetime labor endowments, and not labor earnings.

Effective labor units from the model maps into hourly wage rates from the PSID data.

Our panel data do not allow us to observe the complete lifecycle of earnings for each household, thus we take a multi-step procedure to determining lifetime income for each household.  First, we impute hourly ages for years in each household's lifecycle that are not observed in the PSID.  To do this, we estimate the following equation, separately by household type (where household type is single male, single female, couple with male head, or couple with female head) :

\begin{equation}
ln(w_{i,t}) = \alpha_{i} + \beta_{1}age_{i,t} + \beta_{2}age_{i,t}^{2} + \beta_{3}*age_{i,t}^{3}+interactions with race and educaiton + \varepsilon_{i,t}
\end{equation}

When estimating this equation we only use non-zero observations of wages.

\bibliography{cons_calib_bib}

\end{document}